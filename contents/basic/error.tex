\begin{savequote}[45mm]
\ascii{Any fool can write code that a computer can understand. Good programmers write code that humans can understand.}
\qauthor{\ascii{- Martin Flower}}
\end{savequote}

\chapter{错误处理} 
\label{ch:error-handle}

\begin{content}

\end{content}

\section{错误处理}

\begin{content}

\subsection{错误表示}

\begin{nodiff}{接口:error}
 \begin{go}
package builtin

type error interface {
	Error() string
}
 \end{go}
\end{nodiff}

\subsection{实现1:逻辑错误}

\begin{nodiff}{包:errors}
 \begin{go}
package errors

func New(text string) error {
	return &errorString{text}
}

type errorString struct {
	s string
}

func (e *errorString) Error() string {
	return e.s
}
 \end{go}
\end{nodiff}

\begin{nodiff}{包:errors}
 \begin{go}
package fmt

import "errors"

func Errorf(format string, a ...interface{}) error {
	return errors.New(Sprintf(format, a...))
}
 \end{go}
\end{nodiff}

\subsection{实现2:系统错误}

\begin{nodiff}{包:io}
 \begin{go}
package syscall

type Errno uintptr

const (
	E2BIG           = Errno(0x7)
	EACCES          = Errno(0xd)
	EADDRINUSE      = Errno(0x62)
	EADDRNOTAVAIL   = Errno(0x63)
	EADV            = Errno(0x44)
	EAFNOSUPPORT    = Errno(0x61)
	EAGAIN          = Errno(0xb)
    // ...
)

var errors = [...]string{
	1:   "operation not permitted",
	2:   "no such file or directory",
	3:   "no such process",
	4:   "interrupted system call",
	5:   "input/output error",
    // ...
}

func (e Errno) Error() string {
	if 0 <= int(e) && int(e) < len(errors) {
		if s := errors[e]; s != "" {
			return s
		}
	}
	return "errno " + itoa(int(e))
}
 \end{go}
\end{nodiff}

\end{content}
